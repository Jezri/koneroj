
\documentclass{tufte-book}
\usepackage{titling}
\usepackage{fancyhdr}
\pagestyle{fancy}
\fancyhf{}
\rhead{\thesection}
\lhead{}
\rfoot{\color{titlecolor}{ Page \thepage}}
\usepackage[parfill]{parskip}
\usepackage{verbatim}
\usepackage{mystyle}
\usepackage{natbib}
\usepackage{amsmath}
\usepackage{longtable}
\everymath{\displaystyle}
\providecommand{\tightlist}{%
  \setlength{\itemsep}{0pt}\setlength{\parskip}{0pt}}
\begin{document}
\hypertarget{introductory-analysis}{%
\section{Introductory Analysis}\label{introductory-analysis}}\hypertarget{proof-techniques}{%
\section{Proof Techniques}\label{proof-techniques}}

\hypertarget{proof-of-uniquesness}{%
\subsection{Proof of uniquesness}\label{proof-of-uniquesness}}

Assume two seperate elements with the requiered property and than proove
that they must be equal to eachother.

\hypertarget{proof-by-contradiction}{%
\subsection{Proof by contradiction}\label{proof-by-contradiction}}

Assume that a false stament is true and show that this results in a
contradiction\hypertarget{reals-1}{%
\section{Reals 1}\label{reals-1}}\hypertarget{axioms-of-the-reals}{%
\subsection{Axioms of the reals}\label{axioms-of-the-reals}}

The real numbers are a field which also has ordered and diedekin
complete

\begin{description}
\tightlist
\item[Field]
A set \(F\) of with two opperators \(+\) and \(\cdot\) such that
\(\forall f_1, f_2, f_3 \in F\)
\end{description}

\begin{itemize}
\tightlist
\item
  Addition

  \begin{itemize}
  \tightlist
  \item
    \textbf{(A1)} \((f_1+f_2)+f_3 = f_1 + (f_2 + f_3)\) Addition is
    associative
  \item
    \textbf{(A2)} \(f_1+f_2 = f_1+f_2\) Addition is comutitive
  \item
    \textbf{(A3)} \(\exists 0 | f_1 + 0 = f_1\) Indentity of addition
  \item
    \textbf{(A3)} \(\exists -f_1 | f_1 + (-f_1) = 0\)
  \end{itemize}
\item
  Multiplication

  \begin{itemize}
  \tightlist
  \item
    \textbf{(M1)} \((f_1 f_2) f_3 = f_1 (f_2 f_3)\) Multiplication is
    associative
  \item
    \textbf{(M2)} \(f_1 f_2 = f_1 f_2\) Multiplication is comutitive
  \item
    \textbf{(M3)} \(\exists 1 \neq 0 | f_1 \dot 1 = f_1\) Indentity of
    addition
  \item
    \textbf{(M4)}
    \((\forall f_1 \in F | f_1 \neq 0) \exists f_1^{-1} | f_1 ( f_1^{-1}) = 1\)
  \end{itemize}
\item
  Distrabution

  \begin{itemize}
  \tightlist
  \item
    \textbf{(D1)} \(f_1(f_2+f_3)=f_1f_2 + f_1f_2\) multiplication is
    distributive over addition
  \end{itemize}
\end{itemize}

\textsc{Axioms of comparison}

\begin{itemize}
\tightlist
\item
  Order\footnote{Let \(r_1 r_2 \in \mathbb{R}\) then \(r_1\) is called
    larger than \(r_2\) written \(r_1 > r_2\) if \(r_1 - r_2 > 0\) }

  \begin{itemize}
  \item
    \textbf{(O1)} The trichotomy property of reals.

    \begin{itemize}
    \tightlist
    \item
      \(\left( \forall r \in \mathbb{R} \right)\) exactly one is true
    \item
      \(a > 0 \quad a = 0 \quad a < 0\)
    \end{itemize}
  \item
    \textbf{(O2)} \(a > 0 \wedge b > 0 \quad \to \quad a + b >0\)
  \item
    \textbf{(O3)} \(a > 0 \wedge b > 0 \to ab >0\)
  \end{itemize}
\end{itemize}

\textsc{Diedekin completeness axiom}

\begin{itemize}
\tightlist
\item
  Completeness

  \begin{itemize}
  \tightlist
  \item
    \textbf{(C)} All bounded sets have superium and infamium
  \end{itemize}
\end{itemize}Find the defintion of a monotonic transformation

\hypertarget{collararies}{%
\subsection{Collararies}\label{collararies}}

\begin{enumerate}
\def\labelenumi{\alph{enumi})}
\setcounter{enumi}{1}
\tightlist
\item
  \(a < b \wedge b < c \to a < c\quad\) transitivty\footnote{sort of}
\end{enumerate}

\textsc{Proof}\footnote{\textsc{Definitions}

  \begin{itemize}
  \tightlist
  \item
    \(r_1 \geq b \Leftrightarrow a > b \wedge a = b\)
  \item
    \(a \leq b \Leftrightarrow a = b \wedge a < b\)
  \end{itemize}}

\begin{quote}
\(b > a \Leftrightarrow b-a \geq 0\) \textbf{(i)}

\(b -c \geq 0\) \textbf{(ii)}

combining i and ii with previous theorem \((b-a) + c - b \geq 0\)
\(a - c \geq 0\) (addition axioms) \(c \geq 0\)
\end{quote}

\textsc{Proof} by contradiction
\(a > 0 \Leftrightarrow a^{-1} > 0 \wedge a < 0 \Leftrightarrow a^{-1} < 0\)

\begin{quote}
\textbf{Case} \(a > 0\)

\begin{quote}
\textbf{(i)} \(a \neq 0\) \textbf{Trichotomy}\\
\textbf{(ii)} \(\exists a^{-1}\) \textbf{(M3)} \(\wedge\) \textbf{(i)}

\textbf{Case} \(a^{-1} < 0\)

\begin{quote}
\begin{enumerate}
\tightlist
\item
  \(-a^{-1} > 0\) \textbf{(a)}
\item
  \(-a^{-1} \cdot a > 0\) \textbf{(O3)}
\item
  \(-1 > 0\) \textbf{Contradiction}
\end{enumerate}
\end{quote}

\textbf{Case} \(a^{-1} = 0\) \textgreater{} \#. \(1 = a \cdot a^{-1}\)
\textbf{(M3)} \textgreater{} \#. \(= 0\) \textbf{(Multiplication by
zero)} \textbf{Contradiction}

\textbf{Conculsion} \(a > 0 \Leftrightarrow a^-1 >0\)
\end{quote}

\textbf{Case} \(a < 0\)

\begin{quote}
\textbf{(i)} \(a \neq 0\) \textbf{Trichotomy}\\
\textbf{(ii)} \(\exists a^{-1}\) \textbf{(M3)} \(\wedge\) \textbf{(i)}

\textbf{Case} \(a^{-1} = 0\) \textgreater{} \#.
\(1 = a \cdot a^{-1} = 0\) \textbf{Contradiciton}
\textbf{(Multiplication by zero)} \textbf{(M3)}

\textbf{Case} \(a^{-1} > 0\) \textgreater{} \#. \(-1 = -a a^{-1} > 0\)
\end{quote}
\end{quote}

\begin{quote}
\(1 > 0\) \(a \neq 0 \to a^2 > 0\) \(use this to show 1 > 0\)
\end{quote}\hypertarget{absolute-value}{%
\subsection{Absolute value}\label{absolute-value}}

\textbf{Absolute value}

\(|a| \forall a \in \mathbb{R} := \begin{cases} a \geq 0 \quad = a \\ a < 0 \quad -a \end{cases}\)

\textsc{Properties}

\begin{itemize}
\item
  \(|a| = 0 \Leftrightarrow a = 0\)

  \begin{itemize}
  \tightlist
  \item
    if \(a = 0 \to |a| = 0\) deffinition
  \item
    if \(|a| = 0 \to a = 0\) contrapositive
  \end{itemize}
\item
  \(b |-a | = |-a|\)
\item
  use each three cases of dichotomy
\item
  try d and e tutorial questions
\end{itemize}

\hypertarget{trainagle-inequality}{%
\subsection{Trainagle inequality}\label{trainagle-inequality}}

\(||a|+|b|| \geq |a| + |b|\)

\begin{itemize}
\tightlist
\item
  \(|a + b| \leq |a| + |b| tut\)
\item
  \(| |a| - |b| \leq |a + b|\)
\end{itemize}

proof

\begin{quote}
\begin{itemize}
\tightlist
\item
  \(a = a - b + b \to |a| = |a - b + b| = |a-b| + |b|\)
\item
  \(b = b - a + a \to |b| = |b - a + a| = |b-a| + |a|\)
\item
  by c \(|b| \leq |a - b| + |a| \to |b| \leq |b - a|\)
\item
  combining
\item
  multiply ** \(-1\) **
\item
  \(|a - b| \leq |a| - |b| - |a - b| \leq |a| - |b| \leq a-b\)
  \textbar{} by \(d\)
\item
  \(\Leftrightarrow\) triangle inequility to proove with minus just sub
  in \(-b\)
\end{itemize}
\end{quote}\hypertarget{posittive-square-root}{%
\subsection{Posittive Square Root}\label{posittive-square-root}}

\begin{quote}
\(\forall (r \in \mathbb{R} | r \geq 0 ) \quad \exists \left( r^{\frac{1}{2}} \in \mathbb{R} | r^{\frac{1}{2}} \geq 0 \wedge\, \left(a^{\frac{1}{2}} \right)^2 = a \right)\)
and \(r^{\frac{1}{2}}\) is unique
\end{quote}

\textsc{Proof}

\begin{itemize}
\item
  \textbf{Case} \(r = 0\)

  \(0^2 = 0 \quad \wedge \quad 0 \geq 0 \quad \wedge \quad 0 \text{ is unique} \quad \Rightarrow \quad r^{\frac{1}{2}} = 0\)
\item
  \textbf{Case} \(r > 0\)

  \begin{itemize}
  \tightlist
  \item
    \(\exists\, r^{\frac{1}{2}}\)

    \begin{itemize}
    \tightlist
    \item
      \(\exists S_s \, | \, S_s = \text{sup} S\) \textbf{(lemma 1, lemma
      2, C)}\footnote{\(S = \{ x \in \mathbb{R} \, | \, 0 <x, x^2 < 0 \}\)}
    \item
      \(\text{sup}S^2 = r\) \textbf{(lemma3)}
    \end{itemize}
  \item
    \(r^{\frac{1}{2}}\) is unique\textbf{(lemma 4)}
  \end{itemize}
\end{itemize}

\begin{quote}
\textbf{lemma 1} \(S \neq \emptyset\)

\begin{itemize}
\tightlist
\item
  \textbf{Case} \(a < 1\)

  \begin{itemize}
  \tightlist
  \item
    \(0 < a , a^2 < a \Rightarrow a \in S\)\\
  \end{itemize}
\item
  \textbf{Case} \(a \geq 1\)

  \begin{itemize}
  \tightlist
  \item
    \(0 < \frac{1}{2}, \frac{1}{2}^2 < a \Rightarrow \frac{1}{2} \in S\)
  \end{itemize}
\end{itemize}

\begin{center}\rule{0.5\linewidth}{\linethickness}\end{center}
\end{quote}

\begin{quote}
\textbf{lemma 2} \(S\) is bounded

\begin{itemize}
\tightlist
\item
  \(a+1\) is a bound

  \begin{itemize}
  \tightlist
  \item
    \((a + 1)^2\) is an upperbound of \(S\)

    \begin{itemize}
    \tightlist
    \item
      \(x > a+1 \Rightarrow x^2 > a^2 + 2a + 1 > 2a > a \Rightarrow x \nin S\)
    \end{itemize}
  \end{itemize}
\end{itemize}

\begin{center}\rule{0.5\linewidth}{\linethickness}\end{center}
\end{quote}

\begin{quote}
\textbf{lemma 3} \(S_s^2 = r\)

Assume \(S_s^2 \neq r\)

\begin{itemize}
\tightlist
\item
  \textbf{Case} \(S_s^2 < a\)

  \begin{itemize}
  \tightlist
  \item
    \textbf{Contradiction} \(S_s + \epsilon \in S\) for some
    \(\epsilon > 0\) \textbf{(lemma 3.1)}
  \end{itemize}
\item
  \textbf{Case} \(S_s^2 < 0\)

  \begin{itemize}
  \item
    \textbf{Contradiction}

    \(S_s - \epsilon\) is an upperbound of \(S\) for some
    \(\epsilon > 0\) \textbf{(lemma 3.2)}
  \end{itemize}
\end{itemize}

\begin{center}\rule{0.5\linewidth}{\linethickness}\end{center}
\end{quote}

\begin{quote}
\textbf{lemma 3.1}

\begin{itemize}
\tightlist
\item
  Let \(\epsilon = \text{min} \left\{ \frac{r-S_s}{4S_s}, S_s\right\}\)
\end{itemize}

\begin{longtable}[]{@{}rl@{}}
\toprule
\endhead
\begin{minipage}[t]{0.36\columnwidth}\raggedleft
\((S_s + \epsilon)^2 - a\)\strut
\end{minipage} & \begin{minipage}[t]{0.58\columnwidth}\raggedright
\(=S_s^2 -a + 2 (S_s\epsilon + \epsilon) \epsilon\)
\(\leq S_s - a + (2 S_s + S_s)\frac{r-S_s^2}{4S_s}\)\strut
\end{minipage}\tabularnewline
\begin{minipage}[t]{0.36\columnwidth}\raggedleft
\strut
\end{minipage} & \begin{minipage}[t]{0.58\columnwidth}\raggedright
\(\leq \frac{1}{4}(S_s ^2 -r) < 0\)\strut
\end{minipage}\tabularnewline
\bottomrule
\end{longtable}

\begin{itemize}
\tightlist
\item
  \((S_s + \epsilon)^2 < a\)
\end{itemize}

\begin{center}\rule{0.5\linewidth}{\linethickness}\end{center}
\end{quote}

\begin{quote}
\textbf{lemma 3.2}

\begin{itemize}
\tightlist
\item
  Let \(\epsilon = \frac{S_s^2 -r}{2S_s}\)
\end{itemize}

\begin{longtable}[]{@{}rl@{}}
\toprule
\endhead
\((S_s - \epsilon)^2 - a\) &
\(= S_s^2 - 2 S_s \epsilon - \epsilon^2 -a\)\tabularnewline
& \(> S_s^2 - 2 S_s \epsilon\)\tabularnewline
& \(> 0\) \textbf{(Sub in \(\epsilon\) and solve)}\tabularnewline
\((S_s-\epsilon)^2\) & \$\textgreater{} a\tabularnewline
\bottomrule
\end{longtable}

\begin{itemize}
\tightlist
\item
  \textbf{Contradiction} \(S_s- \epsilon\) is an upperbound less than
  \(S_s\)
\end{itemize}

\begin{center}\rule{0.5\linewidth}{\linethickness}\end{center}
\end{quote}

\begin{quote}
\textbf{lemma 4}

\textbf{(i)} \(x^2 = r \quad y^2 = r \quad \wedge \quad r_1, r_2 > 0\)

\textbf{(ii)} \(\, 0 = r - r = x^2 - y^2 = (x+y)(x-y)\)

\textbf{(iii)} \(x + y >0\) \textbf{(i)}

\textbf{(iv)}\(x-y = 0\) \textbf{(iii, ii)}

\textbf{Conclusion} \(x=y\) So the posisitve square root is unique
\textbf{(iv)}

\begin{center}\rule{0.5\linewidth}{\linethickness}\end{center}
\end{quote}\begin{quote}
\textbf{Peano Axioms of the natural numebrs}

There is a unique subset of of the natural numbers satisfying the
following properties.

\begin{itemize}
\tightlist
\item
  \textbf{P1} \(\,0 \in \mathbb{N}\)
\item
  \textbf{P2} \(a \in \mathbb{N} \Rightarrow a+1 \in \mathbb{N}\)
\item
  \textbf{P3} \(\forall a \in \mathbb{N}, a+1 \neq{0}\)
\item
  \textbf{P4} these porperties uniquely define the natural numbers
\end{itemize}
\end{quote}

Theorem 1.14 are the basic properties of induction * Proof of 1 is done
badly fix * Proof 3 induction is on n not m

write induction in three seperate paragraphs archamedian properties are
important

Learn all named thoerems!

\begin{itemize}
\tightlist
\item
  base case
\item
  inductive step
\item
  proof
\end{itemize}

\hypertarget{the-sets-q-and-the-set-are-both-dense}{%
\section{\texorpdfstring{The sets \mathbb Q and the set
\mathbb{R}\mathbb[Q] are both
dense}{The sets Q and the set are both dense}}\label{the-sets-q-and-the-set-are-both-dense}}

x y in R x\textless{}y -\textgreater{} y-x \textgreater{} 0
\textgreater{} q / y-x \textgreater{} 0 by archamedian principle
\eixts n\_0 \in \mathbb{N} such that n\_o \textgreater{} 2/y-x
\textgreater{} 0

Let S = m \in Z m0 x \textless{} m By AP this is not empty So S is
bounded below So s has a minalmal element by theoerm 1.15\hypertarget{glossary}{%
\section{Glossary}\label{glossary}}

\begin{description}
\tightlist
\item[Bound]
The bound of a set is a number such that all numbers in the set are less
than or greater than the set. A bound is not unique.
\item[Supremum]
The supereme or infamium of a set the least upper bound or the greatest
lower bound and is unique
\item[Maximum and miniumium]
If the superium or infaium are elements of the set than they are the max
and miuminium otherwise there is no maxium or mimiumum
\end{description}\textbf{Absolute value}

\(|a| \forall a \in \mathbb{R} := \begin{cases} a \geq 0 \quad = a \\ a < 0 \quad -a \end{cases}\)\end{document}