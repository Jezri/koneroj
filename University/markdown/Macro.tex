\PassOptionsToPackage{unicode=true}{hyperref} % options for packages loaded elsewhere
\PassOptionsToPackage{hyphens}{url}
%
\documentclass[twocolumn]{article}
\usepackage{lmodern}
\usepackage{amssymb,amsmath}
\usepackage{ifxetex,ifluatex}
\usepackage{fixltx2e} % provides \textsubscript
\ifnum 0\ifxetex 1\fi\ifluatex 1\fi=0 % if pdftex
  \usepackage[T1]{fontenc}
  \usepackage[utf8]{inputenc}
  \usepackage{textcomp} % provides euro and other symbols
\else % if luatex or xelatex
  \usepackage{unicode-math}
  \defaultfontfeatures{Ligatures=TeX,Scale=MatchLowercase}
\fi
% use upquote if available, for straight quotes in verbatim environments
\IfFileExists{upquote.sty}{\usepackage{upquote}}{}
% use microtype if available
\IfFileExists{microtype.sty}{%
\usepackage[]{microtype}
\UseMicrotypeSet[protrusion]{basicmath} % disable protrusion for tt fonts
}{}
\IfFileExists{parskip.sty}{%
\usepackage{parskip}
}{% else
\setlength{\parindent}{0pt}
\setlength{\parskip}{6pt plus 2pt minus 1pt}
}
\usepackage{hyperref}
\hypersetup{
            pdfborder={0 0 0},
            breaklinks=true}
\urlstyle{same}  % don't use monospace font for urls
\setlength{\emergencystretch}{3em}  % prevent overfull lines
\providecommand{\tightlist}{%
  \setlength{\itemsep}{0pt}\setlength{\parskip}{0pt}}
\setcounter{secnumdepth}{0}
% Redefines (sub)paragraphs to behave more like sections
\ifx\paragraph\undefined\else
\let\oldparagraph\paragraph
\renewcommand{\paragraph}[1]{\oldparagraph{#1}\mbox{}}
\fi
\ifx\subparagraph\undefined\else
\let\oldsubparagraph\subparagraph
\renewcommand{\subparagraph}[1]{\oldsubparagraph{#1}\mbox{}}
\fi

% set default figure placement to htbp
\makeatletter
\def\fps@figure{htbp}
\makeatother


\date{}

\begin{document}

\hypertarget{the-classical-school}{%
\section{The Classical School}\label{the-classical-school}}

\hypertarget{the-production-function}{%
\subsection{The production function}\label{the-production-function}}

\begin{quote}
\(Y=F(K,N)\) In the short run labour is said to be fixed. Where K is
capital and N is labour. So output varies only with labour drawn from a
fixed population size. We mostly look at diminishing marginal product to
labour.
\end{quote}

To get the demand for labour. Multiply the product of labour by the
price of the products in the market. This gives the total benifit to a
producer of hiring additional workers. Subtact from this the costs or
wages of labour. Then differentaite to find the point of maximium
profit.

Or MC for a firm is \(\frac{W}{MPN}\)

In a perfectly competitive market. \(P=MC=\frac{W}{MPN}\) So
\(\frac{W}{P} = MPN\) ie The number of goods for which the last labourer
is paid the full value of goods they work produce is the whole number of
goods they produce? Wages overprices is the real wage rate Macro
Economics ===============

\hypertarget{gdp}{%
\subsection{GDP}\label{gdp}}

\begin{description}
\item[GDP(Gross Domestic product)]
The Total amount of goods and services produced within an economy in a
given year {[}\^{}mn{]} \{-\}There are three ways of culculating this *
Expenditure This must only include expenditure on goods and services
produced within the economy (no imports, and no goods produced in a
previous year) * Income This must only use income obtained by selling
goods and services (no transfer payments) * Output
\end{description}

\hypertarget{gdp-composition}{%
\section{GDP composition}\label{gdp-composition}}

To measure the GDP\footnote{GDP and total demand(Z) are used
  interchangably} it is simplest to measure the amount spent on goods
and services and then subtract the part of that which is spent on goods
and services produced outside the economy (imports) or before the given
year (invetories). Finaly goods not bought in the bought elsewhere
(exports) or stored for the future are added.\footnote{Exports and
  inventories are ignored in the begining part of the course}

\begin{itemize}
\tightlist
\item
  Consumption(C): The goods and services purchased by consumers
\item
  Investment(I): The sum of

  \begin{itemize}
  \tightlist
  \item
    no-resedential investment: Capital equipment and land bought by
    firms
  \item
    resedntial investment: Housing bought by consumers
  \end{itemize}
\item
  Goverment spending(G): The amount the goverment spendings buying goods
  and services from firms and employing workers. (goverment tranfers are
  not payments for work done and are not included)
\item
  Net exports (X-I): The total amount of exports minus imports.
\item
  Net inventory build up
\end{itemize}

This brings us to the equation \(Z = C + I + G\)

\hypertarget{consumption}{%
\subsection{Consumption}\label{consumption}}

Consumption is a function of disposable income \footnote{income minus
  taxation} (\(Y_D\)) \[ C(Y_D) \] Unemployment -------------

\hypertarget{inflation}{%
\subsection{Inflation}\label{inflation}}

\hypertarget{philips-curve}{%
\subsection{Philips curve}\label{philips-curve}}

\hypertarget{gdp-composition-2}{%
\section{GDP composition 2}\label{gdp-composition-2}}

Go over chapter 2

Net foriegn factor income.

Indirect taxes :Sin taxes , value add tax , import taxes

\begin{description}
\tightlist
\item[Directs taxes]
Direct on factor imput, wages profit
\end{description}

GDP at market price - direct taxs +(net subsidies)\footnote{indirect
  taxes - subsidies}

\hypertarget{further-adjustments}{%
\subsection{further adjustments}\label{further-adjustments}}

\begin{itemize}
\tightlist
\item
  further transaction on household income
\item
  Insurance contrabutions(money is taken directly taken, south african
  pensions come directly from tax not from fund)
\item
  Unemployment funds (are in douth africa)
\item
  Corporaate taxes
\item
  Profits that could have been paid by firms that are retained by firms
\item
  transfer payments\footnote{Do not confuse payments to and from
    unemployment and pension payments} This results in personal income
\item
  taxes on interest This results in disposable income : The amount of
  income a consumer can produce
\end{itemize}

GDP is concerned with the amount of production that takes place in a
country GNP is by national citizen

GDP + income from foriegn source - production from foriegn sources

Output(Value add) = Output(Income) + assume not corporate profit is
retained.

Output(Value added) = output(expenditure) + No inventories

Output(expenditure = output(income) + No saving

\textbf{Important}

\begin{description}
\tightlist
\item[Nominal vs real GDP]
Nominal GDP = real GDP * current prices
\end{description}

\begin{itemize}
\tightlist
\item
  Prices measured as a pecentage of the bases year
\end{itemize}

Real GDP higher than nominal GDP means increase in output\footnote{Q:
  What is calculated first infaltion or gdp, Why not exponential but go
  over}

\hypertarget{unemplyment-or-inflation}{%
\subsection{Unemplyment or inflation}\label{unemplyment-or-inflation}}

\begin{description}
\tightlist
\item[Strict unemployment]
People that are activly looking for work Broad unemplyment

Poeple activly looking for work plus discouraged works (everybody who
would like to work)
\end{description}

Broad is greater the strict easaly proovable

U or Ut is the number of people unemployed u or ut is the unemployment
rate

\begin{description}
\tightlist
\item[Paticipation rate]
The the labour force over the population size. Higher participation
rates tend to have higher employment rates.
\end{description}

\hypertarget{problems-with-unemplyment}{%
\subsection{Problems with unemplyment}\label{problems-with-unemplyment}}

\begin{itemize}
\tightlist
\item
  GDP excludes the illigal economy and exculdes the legal economy that
  is not reported for tax evasion.
\item
  Good unemployment benefits may couses people to register as
  unemployed.
\item
  Unemployemnt couses less than optimal production.
\end{itemize}

\#inflation An increase in the change of general price levels. inflation
rate is the differentite of inflation. An index may be simple or
compound

CPI is used in south africa (goods consumed by a typical or average
houshold)

\begin{itemize}
\tightlist
\item
  Conducts infequent houshold servays every five or more year to get
  weightings
\item
  Consumer price index
\item
  State SA trakes some prices monthly and others quaterly
\item
  Month by month inflationn a - b / a
\item
  monthly anual inflation rate. jan to jan \ldots{} dec to dec
\item
  annual = average of monthly annual
\end{itemize}

\begin{enumerate}
\tightlist
\item
  find the size of the labour force
\end{enumerate}

\begin{description}
\tightlist
\item[GDP deflator]
Real GDP - Nominal GDP / reaGDP
\end{description}

GDP deflator and CPI move together most of the time but cpi moves faster
from international shokes.

Competition determines how much price shocks are communicated to
consumers.

Hyperinflation and deflation

Inflation affect income distrabution

\begin{itemize}
\tightlist
\item
  Fixed income earners such as pensioners loose income
\item
  Distorions
\item
  Bracket creep(Goverments try to adjust)
\item
  Exchange and inflation tend to move together
\end{itemize}

Is inflation ever good

\begin{itemize}
\tightlist
\item
  In japan moderate inflation could have worked
\item
  High deflation can lead to uncertainty
\item
  Why does low inflation make monetary policy useless

  \begin{itemize}
  \tightlist
  \item
    Inflation and interest rate move together.
  \item
    Centeral bank cannot reduce interest rates below zero
  \end{itemize}
\end{itemize}

\hypertarget{chapter-3}{%
\section{Chapter 3}\label{chapter-3}}

\hypertarget{core-assumptions}{%
\subsection{Core assumptions}\label{core-assumptions}}

\hypertarget{section}{%
\subsection{}\label{section}}

\hypertarget{understanding-the-economics-of-gdp-equation.}{%
\section{Understanding the economics of GDP
equation.}\label{understanding-the-economics-of-gdp-equation.}}

\hypertarget{the-aggregate-expenditure-model.}{%
\subsection{The aggregate expenditure
model.}\label{the-aggregate-expenditure-model.}}

In equalibrium

\begin{itemize}
\tightlist
\item
  \(Y = \text{income} = \text{output}\) \textbf{45 degree line}
\item
  \(Y = C + I + G + (X-M)\)
\item
  solve for \(Y\)
\item
  Alternitive leakages vs injections in the goods/output market
\item
  Find saving (the part of disposable income which is not consumed)

  \begin{itemize}
  \tightlist
  \item
    \(Yd\) is disposable income \(Y -T\) \footnote{Taxes do not very
      with income for simpplity.}
  \end{itemize}
\item
  National income may be veiw as the aount of income earned or the
  amount generated.
\item
  \(Y= C + T + S\)
\item
  Generation

  \begin{itemize}
  \tightlist
  \item
    Generated by private investment goverment of and private spending.
  \item
    Or factor income.
  \end{itemize}
\item
  Equat the right hand side of both equations and solve for S
\item
  Group together goverment policy
\item
  Assume govermant defiicit
\item
  \(S= \text{private savings} \text{corperate saveings}\)
\item
  Some private savings are used to finaces deficit so there is crowding
  out of privated investment.
\item
  Solve for I to show this
\item
  If goverment is running a surplus goverment has savings which is used
  to finace private sector investment.
\item
  \(S_p + S_g = I\)
\item
  Than add G on both sides of the inequality.
\item
  LHS saving and taxes are lelakages
\item
  In equilibrium leakages are equal to injections
\end{itemize}

Diagamatically

\begin{itemize}
\tightlist
\item
  Draw a line represnting the realtionship between I + G and S + T
\item
  replace S by a function of income.
\item
  S + T veries positivly with Y and S +T0 is negative.
\item
  Leakages are equel to investment when the two lines meet
\end{itemize}

\hypertarget{introducing-imports-and-exports}{%
\subsection{Introducing imports and
exports}\label{introducing-imports-and-exports}}

\begin{itemize}
\tightlist
\item
  Exports are a leakage. Impports are an injection
\item
  Solve for S with exports and imports
\item
  group G and T
\item
  group X and M
\item
  Interprete the equation

  \begin{itemize}
  \tightlist
  \item
    assume that G \textgreater{} T and X \textgreater{} M
  \item
    curret and finacial acount
  \item
    Blance of payment meens imports must be equal to 0
  \item
    Balance of payments := current acount + finacial account
  \item
    Part of our saving are being held off shore.
  \item
    \(F_s\) is domestic savings held offshore.
  \item
    Solve for I in an open economy
  \item
    South African rand is volatile becouse of high reaince on short term
    forign loans.
  \end{itemize}
\item
  Methedology used to find equilibrium dependece on veriables given
\end{itemize}

\hypertarget{paradox-of-thrift}{%
\section{Paradox of thrift}\label{paradox-of-thrift}}

A paradox is a seemingly ontradictory stament that may none the less be
true on a deaper level of meaning or understanding.

A household try to save more their income decreases by a level such that
thier income remains unsaved.

\begin{itemize}
\tightlist
\item
  I = S + (T-G)
\item
  I = 1 - c\_0 + (1 - c\_1)Y\_n +(t\_0 + t\_1Y) -G{]}
\item
  assume autonomous savings decreases.
\item
  Draw the graph of the consumption and savings function.
\item
  People tend to spend more than they save.
\item
  Saving functions is flatter than consumtion function
\item
  More savings meens consumption function moves down and savings
  function moves up.
\item
  Change in equilibrium
\item
  So lower income meens less saved even though thier is a higher savings
  rate.
\item
  This is not indefinate the susesiive falls get smaller and smaller.
\item
  If saving and income induced tax revenues shrink.
\item
  Invesment is assumed to be constant. Savings levels will eventually
  equalize.
\item
  In reality output and investment move together.
\item
  Investment is the engine of growth in fast growign economies.
\end{itemize}

\hypertarget{is-the-goverement-omnipontent}{%
\section{Is the goverement
omnipontent}\label{is-the-goverement-omnipontent}}

Goverment cannot change goverment spendign at its own will Medium term
expenditure plan in south africa is the 3 year plan which the goverment
must stick there are also palamentry adjsutments. Large deficits
increase risk which adversly effect exchae rate.

Anticipations are likly to matter, permanent or tempory decrease in
taxes have different effects. Full employment plus stimulation will
could result in inflation. Expansionary fiscal policy may have short run
beneficail effects. Which increase the amount of interest that needs to
be paid.

large defficits can crowd out private investment.

Is it not posible for trusted goverments to get very low interest rate
loans such that interest may even dip below infaltion?

Change in inventories are cuased by a lack of equilibrium

In cahptr 5 we will include finacial markets and in 7 we will include
the differnece between real and nominal interest. Models which show
interaction between the outpt market finacial market and labour markets

The link between the utput market and the finacial market are interest
rates and income Changes in money supply chaneg interest rates Changes
in the interest rate change the output level This again changes income.

\hypertarget{the-mathamatical-model-of-the-finacial-market.}{%
\section{The mathamatical model of the finacial
market.}\label{the-mathamatical-model-of-the-finacial-market.}}

\hypertarget{functions-of-money}{%
\subsection{Functions of money}\label{functions-of-money}}

\hypertarget{motives-fror-holding-money}{%
\subsection{Motives fror holding
money}\label{motives-fror-holding-money}}

\hypertarget{nessesary-information}{%
\subsection{nessesary information}\label{nessesary-information}}

\begin{itemize}
\tightlist
\item
  Assume their are only two markets

  \begin{itemize}
  \item
    Exclude stock market
  \item
    Exclude Diravite market -W is finacial wealth total amount owed +
    total amount of bonds ownd
  \item
    Cu is currency held by public
  \item
    D is the deposits in private /or commercial banks
  \item
    Using simple balance sheets (Aggregated all balance sheets for
    commercial banks)
  \item
    Liability are accounts held by consumers
  \item
    Assets are goverment bnods(assume short term goverment bond with a
    one year period)
  \item
    Bonds are bought from goverment
  \item
    Bonds are an interest yielding good
  \item
    Bond are fixed interst rate
  \item
    Goverment issues bonds to the private sector and to the central bank

    Assets Liabi ----- ------- ---------
  \end{itemize}
\end{itemize}

Required reserve ratio is the ratio of reserves to assets that may be
issued based on the amounts of bonds. Not on loans. Money is created in
the baking sector though loans and lent out through.

Look at differen measures of money supply in south africa. Assume money
earns no interest. Assume all acounts are cheque account deposits rather
than saving accounts IE low interest rates.

\hypertarget{the-inverse-relationship-between-interest-rates-and-the-price-of-bonds.-zero-coupon-bonds.-face-value-is-determined-by-the-issuer-or-goverment.-this-is-the-result-that-the-goverment-borrows-from-consumer}{%
\subsection{The inverse relationship between interest rates and the
price of bonds. Zero coupon bonds. face value is determined by the
issuer or goverment. This is the result that the goverment borrows from
consumer}\label{the-inverse-relationship-between-interest-rates-and-the-price-of-bonds.-zero-coupon-bonds.-face-value-is-determined-by-the-issuer-or-goverment.-this-is-the-result-that-the-goverment-borrows-from-consumer}}

\begin{itemize}
\item
  Assume there is an expectation that interest rates will increase in
  the future. Assume individuals want to hold more money
\item
  Total wealth is money + bonds
\item
  holding more money meens selling bonds
\item
  Increase in the supply of bonds.
\item
  So price of bonds increases
\item
  Ask about this it is confusing.
\item
  To see the effect of interest rate expecations
\item
\item
  Money Demand = currency demand by the private sector + Deposites
\item
  \(\frac{C_u}{D} = c\) so curreny is a proportion of money demanded?
\item
  derive the ration between Md and Dd = 1-c
\item
  h donnates central bank money = currency + deposites with central bank
  (held by private banks)
\item
  GO OVER FULL EQUATIONS FOR MONEY SUPPLY. \# Budget speech Links to
  concepts studied. Go over in tut tests.
\end{itemize}

\hypertarget{monetary-policy-effectivness-and-the-islm-curve}{%
\section{Monetary policy effectivness and the islm
curve}\label{monetary-policy-effectivness-and-the-islm-curve}}

\hypertarget{expansionary-monitry-policy}{%
\subsection{Expansionary monitry
policy}\label{expansionary-monitry-policy}}

\begin{itemize}
\item
  Repo rate
\item
  Easy open market operations
\item
  The horizontal shift in equilibrium income in the islm model is the
  same as the horizontal shift in the aggregate expenditure model
\item
  A verticle IS curve means there is no change of income with a change
  in income
\item
  Find equation for the interest elasticty of income
\item
  Increasing Tax rate and imports reduce the level of output. So small
  m1 and t1 lead to the same effect of big c1 and b1
\end{itemize}

\hypertarget{the-effectivness-of-fiscal-policy-under-different-assumptioms-about-the-slop-of-the-is-function}{%
\subsection{The effectivness of fiscal policy under different
assumptioms about the slop of the IS
function}\label{the-effectivness-of-fiscal-policy-under-different-assumptioms-about-the-slop-of-the-is-function}}

\begin{itemize}
\tightlist
\item
  Expansionary fiscal policy will shift the IS curve outwards
\item
  Imposes the same shift when IS curve is steep or verticle
\item
  Look at the case where the IS function is verticle
\item
  The steeper the IS curve increase interest more (So interest rate
  increases)
\item
  A steeper IS means more effective fiscal poicy
\end{itemize}

\hypertarget{monetary-plolicy-effectivness-lm-slope}{%
\section{Monetary plolicy effectivness (LM
slope)}\label{monetary-plolicy-effectivness-lm-slope}}

\hypertarget{relitivly-flat-lm-curve}{%
\subsection{Relitivly flat LM curve}\label{relitivly-flat-lm-curve}}

\begin{itemize}
\tightlist
\item
  Money demand is unresopsive to changes in income or interest
\item
  Flat M\_d curve results in a flat LM curve
\item
  i = \frac{d_0 -M}{d1}+\frac{d_1}{d_2}Y

  \begin{itemize}
  \tightlist
  \item
    \(Md = d_0 + d_1Y -d_2i\)
  \item
    LM \(Y = \frac{m}{d_1} ...\)
  \item
    A relitivly flat lm schedule low d-2(interest elasticty of money
    demand)

    \begin{itemize}
    \tightlist
    \item
      A high interest elasticty of d\_2 means a flat LM
    \end{itemize}
  \end{itemize}
\item
  Out put changes by a larger amount in the aggregate expenditure model
  that in the islm model

  \begin{itemize}
  \tightlist
  \item
    Try show this graphically
  \item
    Money demand will increase
  \item
    So investment decreases
  \end{itemize}
\item
  Steep Md = steep LM reactive moentary policy
\item
  Verticle Md must be the same as Ms = verticle LM so reactive LM model
\end{itemize}

\hypertarget{fiscal-policy-altering-the-slope-of-the-lm}{%
\section{Fiscal policy altering the slope of the
LM}\label{fiscal-policy-altering-the-slope-of-the-lm}}

\begin{itemize}
\tightlist
\item
  Flatter LM makes fiscal poliy more effective
\item
  An ecpansionsary fiscal poicy leads to an increase in interest rates
  leads to an decrase in investment ``Crowding out''
\item
  Which is consistent with a relitivly flat moeny demand
\item
  Speak about large vs small increase in interest rate relating to large
  vs small changes in investment.
\end{itemize}

\hypertarget{liquidity-trap}{%
\section{Liquidity trap}\label{liquidity-trap}}

\begin{itemize}
\tightlist
\item
  When both the goods and money market are simulationiously in
  eqiulibrium We have the LM cruvre
\item
  Wealth can only be held in money or bonds
\item
  If all wealth is held in the form of money wealth is only in th most
  liqiud asset
\item
  Us economy during the great depession
\item
  Japanese ressesion of the 1990
\item
  During a liquidity trap interest rates are low so bond pirces are
  high.
\item
  Depresion is more intense than a resesion same time but larger fall
\item
  LM function is flat or horizontal
\item
  So there is a flat M\_d d\_2 = intfitiy Money demand flattens out
\item
  LM schedule is the mirror image
\item
  As the central bank increases money supply Will result in an equel
  increase in money demand. interest rates are very low.
\item
  There is a high risk of companies going bankcrupt
\item
  If interest rates are closs to zero they can only move upwards
\item
  So bond prices are expected to decrease
\item
  So capital losses are a large risk
\item
  So fiscal policy is the solution
\end{itemize}

\hypertarget{policy-mix}{%
\section{Policy mix}\label{policy-mix}}

Fiscal policy and monetary policy mix

\hypertarget{resesion}{%
\subsection{Resesion}\label{resesion}}

\begin{itemize}
\tightlist
\item
  Output is very low and we are not in a liquidiy trap

  \begin{itemize}
  \tightlist
  \item
    Use expansionry policy for both
  \end{itemize}
\item
  If output Y\_0 is very low you can use expansionary fiscal policy
  together with expansionary monetary policy. So the interest rate is
  unchaged. So no crowding out
\end{itemize}

\hypertarget{reduce-budget-defict-withouot-an-adverse-effect-on-output.}{%
\subsection{Reduce budget defict withouot an adverse effect on
output.}\label{reduce-budget-defict-withouot-an-adverse-effect-on-output.}}

Contractionsry fiscal policy Expansionary monatary policy

Consider the effect of eogounous veriables

\begin{itemize}
\tightlist
\item
  Autonoumous investment moves IS outwards Same as fiscal shifts
\item
  Credit card meen less moeny so LM shifts downwards?
\item
  Any factors other than income which shift the LM schedule
\end{itemize}

\hypertarget{static-analyisis-is-what-we-have-done-so-far-dynamic-analyisi-would-also-be-useful.}{%
\section{Static analyisis is what we have done so far dynamic analyisi
would also be
useful.}\label{static-analyisis-is-what-we-have-done-so-far-dynamic-analyisi-would-also-be-useful.}}

\begin{itemize}
\tightlist
\item
  Investment is done in dynamics
\item
  Plot each veraible against time
\end{itemize}

The response to chages in disopable income depends on weather the
increase in taxes in temporary or peermanent. A temporary change has
minamal effect Interest rates only stimulate investment if there is no
spare capacity. They also take some time to repond to income changes
Finally look at the resonse of consumtion on emloyment \ldots{}
percentange point is used for changes in percentag

Other veriables

\begin{itemize}
\tightlist
\item
  Lag effect from fedral funds
\item
  relativly small fall
\item
  emplotment is similar to effect on ouput
\item
  Unemployment rate increases
\item
  effect on price level is small deflation over time
\end{itemize}

\textbf{This imply that ISML works}

** Next tut test in 2 weeks**

\hypertarget{summary.}{%
\section{Summary.}\label{summary.}}

Aggregate expenditure rather then demand.

AD prices arn't fixed AE prices are fixed.

\begin{itemize}
\tightlist
\item
  Output determination in the medium run.
\item
  Determined by supply side factors.
\end{itemize}

\hypertarget{labour-market}{%
\section{Labour market}\label{labour-market}}

\hypertarget{output-market-and-labour-market}{%
\subsection{Output market and labour
market}\label{output-market-and-labour-market}}

Aggregtae expaedntiure increase so firms increas output so firms
increase employment so unemployment goes down so wages go up so prices
go up so prices go up so wages go up.

W: Nominal wages or nominal wages

Real wage rate = \(\frac{W}{P}\)

So price wage spiral. Vicouse cycle of increases in money and inflation.

\textbf{Relax the assumption of fixed prices}

The total population is devided into. The instatutional civilian
poputaion and the non instatutional civilian population.

Non instatutional is all individuals less then 16 + the number of people
in prison and the poeple in asylums and the poeple in the miltary and in
old age homes.

Instatutional: Labour force and out of labour force.

Employed: \textbf{Narrow defnition} Full time or part time working
people with some form of renumeration(\textbf{posibly in kind but not
for the time}).

Labour force: employed or activly looking for work while available for
employment.

Out of labour force: Discouraged workers, retired people
\textgreater{}=65yrs, Students in tertairy labour force + severly
disabled people + people who don't wan't to work Discouraged workers:
Would like to work but are not activly looking for work.

\hypertarget{descriptions}{%
\subsection{Descriptions}\label{descriptions}}

\hypertarget{employment}{%
\subsubsection{Employment}\label{employment}}

\hypertarget{unemployment}{%
\subsubsection{Unemployment}\label{unemployment}}

\hypertarget{participation-rate}{%
\subsubsection{Participation rate}\label{participation-rate}}

\hypertarget{wages-price}{%
\subsection{Wages \& Price}\label{wages-price}}

\hypertarget{works-and-unions}{%
\subsubsection{Works and unions}\label{works-and-unions}}

\hypertarget{prices}{%
\subsubsection{Prices}\label{prices}}

\hypertarget{employment-and-output.}{%
\subsection{Employment and output.}\label{employment-and-output.}}

\hypertarget{aggregate-supply-aggregate-demand-function.}{%
\section{Aggregate supply aggregate demand
function.}\label{aggregate-supply-aggregate-demand-function.}}

\hypertarget{look-at-dyanamics-and-rates.}{%
\section{Look at dyanamics and
rates.}\label{look-at-dyanamics-and-rates.}}

\end{document}
