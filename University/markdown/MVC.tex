\PassOptionsToPackage{unicode=true}{hyperref} % options for packages loaded elsewhere
\PassOptionsToPackage{hyphens}{url}
%
\documentclass[]{article}
\usepackage{lmodern}
\usepackage{amssymb,amsmath}
\usepackage{ifxetex,ifluatex}
\usepackage{fixltx2e} % provides \textsubscript
\ifnum 0\ifxetex 1\fi\ifluatex 1\fi=0 % if pdftex
  \usepackage[T1]{fontenc}
  \usepackage[utf8]{inputenc}
  \usepackage{textcomp} % provides euro and other symbols
\else % if luatex or xelatex
  \usepackage{unicode-math}
  \defaultfontfeatures{Ligatures=TeX,Scale=MatchLowercase}
\fi
% use upquote if available, for straight quotes in verbatim environments
\IfFileExists{upquote.sty}{\usepackage{upquote}}{}
% use microtype if available
\IfFileExists{microtype.sty}{%
\usepackage[]{microtype}
\UseMicrotypeSet[protrusion]{basicmath} % disable protrusion for tt fonts
}{}
\IfFileExists{parskip.sty}{%
\usepackage{parskip}
}{% else
\setlength{\parindent}{0pt}
\setlength{\parskip}{6pt plus 2pt minus 1pt}
}
\usepackage{hyperref}
\hypersetup{
            pdfborder={0 0 0},
            breaklinks=true}
\urlstyle{same}  % don't use monospace font for urls
\setlength{\emergencystretch}{3em}  % prevent overfull lines
\providecommand{\tightlist}{%
  \setlength{\itemsep}{0pt}\setlength{\parskip}{0pt}}
\setcounter{secnumdepth}{0}
% Redefines (sub)paragraphs to behave more like sections
\ifx\paragraph\undefined\else
\let\oldparagraph\paragraph
\renewcommand{\paragraph}[1]{\oldparagraph{#1}\mbox{}}
\fi
\ifx\subparagraph\undefined\else
\let\oldsubparagraph\subparagraph
\renewcommand{\subparagraph}[1]{\oldsubparagraph{#1}\mbox{}}
\fi

% set default figure placement to htbp
\makeatletter
\def\fps@figure{htbp}
\makeatother


\date{}

\begin{document}

\newcommand{\R}{\mathbb{R}}

\hypertarget{linearity}{%
\section{Linearity}\label{linearity}}

A function f(x) is linear (in x) if f(ax +by ) = af(x) + bf(y)
\(\forall x,y \in a\) and \(a,b\) scalars or real * The curl disrabutes
over the sum proof by linearity of partial derivates \textbf{The sum of
two linear functions id linear} *
\(\nabla \times (g\vec{F}) = (\nabla g)\times \vec{F} + g(\nabla \times \vec{F})\)
* \textbf{g} \$ \vec{F \times G} = ()\nabla \times \vec{F}
)\cdot \vec{G}- (\nabla \times \vec{G})\cdot \vec{F}

\begin{itemize}
\tightlist
\item
  Show that
  \(\nabla \cdot (g\nabla f) = (\nabla g) \cdot (\nabla f) + g\nabla ^2 f\)
  **Let \(\nabla f = \vec{F}\)
\item
  Most preceding theorems are prooved of proceeding ones rather than
  from first principles.
\item
  Revise definitions and properties of section 1.2 !!
\end{itemize}

\hypertarget{second-order-partial-derivatives-with-respect-to-multiple-veriable}{%
\section{Second order partial derivatives with respect to multiple
veriable}\label{second-order-partial-derivatives-with-respect-to-multiple-veriable}}

\begin{itemize}
\tightlist
\item
  Let \$f:\mathbb{R}\^{}n \to \mathbb{R}
\item
  Than
  \(\frac{\partial ^2f}{\partial k_x \partial x_j} \overbrace{mostly}{=} \frac{\partial ^2 f}{\partial x_k \partial x_j}\)
  \textbf{Problems arise for example when some of the limits don't
  exist}
\end{itemize}

\hypertarget{chain-rule}{%
\section{Chain Rule}\label{chain-rule}}

Partially derive in terms of x\_1 to x\_n before pluging in x = g(t)
Onfriday

Use l'hoppital for dirrectional derivatives

\hypertarget{tangents-and-normals}{%
\section{Tangents and normals}\label{tangents-and-normals}}

\hypertarget{cuts}{%
\subsection{cuts}\label{cuts}}

We get cross sections 1 dimension less than the perant function
\(S \{} \vec{x} \in \mathbb{R}^n : f(x) = c \}\) this can give curves or
filled spaces and

\begin{itemize}
\item
  \$\vec{x} \in S is a regular point if \(\grad{F} (\vec{x}) \neq )\)
\item
  \$ is a singular point if \$\grad{f}(\vec{x})=0; If grad is a tangent
  zero grads make things wierd.
\item
  Let S be a hyper surface in \$\mathbb{R}\^{}n. than n is a noraml to s
  at x\_0 is \(\forall \gamma :\mathbb{R} \to \mathbb{R}^n\) such that

  \begin{itemize}
  \tightlist
  \item
    \(\gamma(t) \in S\)
  \item
    \(\gamma(t_0) = x_0 for some t_0 in \mathbb{R}\)
  \item
    \(\vec{n} \cdot \gamma'(t) = 0\)
  \end{itemize}
\end{itemize}

\(\grad{d}(\vec{x_0}) \cdot \vec{\gamma}'(t_0) = 0\) Every directional
derivative is zero.

Proof \(\forall \gamma R \to R^n | gamma t_0 = x_0 for some t_0 \in R\)
and gamma(t) \in S forall t in R we have f(gamma t = c), so that (f
(\gamma  t )`= 0) \Rightarrow = f'(gamma t) gamma'(t) = 0 \Rughtarrow =
\grad f (gamma t) \cdot gamma't =0

in paticular at t0: \gamma(t\_0) = x\_a and
\((\grad{f}(gamma(t0)) = \grad{f_0}, \grad{f}(x_0) \cdot \gamma(t_0) = 0\)

1.5.4 The set of tangent vectors to a hyper surface x at x zero is the
set t at x\_0 is all tangent vectors of cureve on the surface that are
going through x0

\hypertarget{greens-theorems}{%
\section{Green's theorems}\label{greens-theorems}}

\$ Let D be a region in \(\mathbb{R}\)

\end{document}
