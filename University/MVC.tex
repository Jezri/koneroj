
\documentclass{tufte-book}
\usepackage{titling}
\usepackage{fancyhdr}
\pagestyle{fancy}
\fancyhf{}
\rhead{\thesection}
\lhead{}
\rfoot{\color{titlecolor}{ Page \thepage}}
\usepackage[parfill]{parskip}
\usepackage{verbatim}
\usepackage{mystyle}
\usepackage{natbib}
\usepackage{amsmath}
\usepackage{longtable}
\everymath{\displaystyle}
\providecommand{\tightlist}{%
  \setlength{\itemsep}{0pt}\setlength{\parskip}{0pt}}
\begin{document}
\hypertarget{multi-veriable-culculus}{%
\section{Multi Veriable Culculus}\label{multi-veriable-culculus}}

\hypertarget{vector-function}{%
\subsection[Vector function]{\texorpdfstring{Vector function\footnote{Row
  and column vectors are treated interchangably in the coarse}}{Vector function}}\label{vector-function}}

\(\vec{F} :: R^n \rightarrow R^m\)

\(\text{Domain} \quad \subseteq R^n\)

\(\text{Range} \quad \subseteq R^m\)

\hypertarget{continuity}{%
\subsection[Continuity]{\texorpdfstring{Continuity\footnote{All
  functions in the coarse are continues}}{Continuity}}\label{continuity}}

\(\vec{F}\) is continues \(\forall \vec{a}\) if the limit exists and and
is equal to the function for all a\$

Limit:

The limit of of a vector function \(\vec{F}(\vec{X})\) is defined as

\(\lim_{ \vec{x} \to \vec{a} } \vec{F} (\vec{x} ) = \vec{l}\)

if for each \(\epsilon > 0 \exists \delta > 0\)

such that \(\quad 0< ||\vec{x}-\vec{a}|| < \delta\)

\(\Leftrightarrow || \vec{F} (\vec{l})|| < \epsilon\)

\hypertarget{scalar-triple-product}{%
\subsection{Scalar triple product}\label{scalar-triple-product}}

\(\vec{x} \cdot (\vec{y} \times \vec{z}) =  \begin{vmatrix}  x_1 & x_2 & x_3 \\  y_1 & y_2 & y_3 \\  z_1 & z_2 & z_3 \\  \end{vmatrix}\)

\hypertarget{projection}{%
\subsection{Projection}\label{projection}}

\(\text{proj}_a b = \frac{a \cdot b}{|a|^2} a\quad\)

\hypertarget{componenets}{%
\subsection{Componenets}\label{componenets}}

\(\frac{a \cdot b}{|a|}\)

Do shwats and traingle inequilty

Parametric form:

An equation is in parametric form if it is of the type
\(\vec{F} :: \mathbb{R} \to \mathbb{R}^n\)

for a circle:
\[f(t) =  \begin{pmatrix} r \cos (t) \\ r \sin (t) \end{pmatrix}\]

Continuity:

\(\vec{F} :: \mathbb{R} \to \mathbb{R}^n = \{f_i(t)\}_{i=1}^{n} \quad | \forall i \in [1 , n] \; f_i:: \mathbb{R} \to \mathbb{R}\)

\(\vec{F}\) is continious \(\iff\) \(\forall i \in [1,n] \; f_i\) is
continious\hypertarget{differentiates-of-vector-functions}{%
\section[Differentiates of vector
functions]{\texorpdfstring{Differentiates of vector functions\footnote{\textbf{Definition}
  \(:=\) (is defined as)}}{Differentiates of vector functions}}\label{differentiates-of-vector-functions}}

\hypertarget{differentiates-of-vecf-mathbbr-to-mathbbrn}{%
\subsection{\texorpdfstring{Differentiates of
\(\vec{F}: \mathbb{R} \to \mathbb{R}^n\)}{Differentiates of \textbackslash{}vec\{F\}: \textbackslash{}mathbb\{R\} \textbackslash{}to \textbackslash{}mathbb\{R\}\^{}n}}\label{differentiates-of-vecf-mathbbr-to-mathbbrn}}

\begin{longtable}[]{@{}ll@{}}
\toprule
RHS & LHS\tabularnewline
\midrule
\endhead
\(\vec{F}'\) &
\(=\lim_{a \to 0}\left( \frac{1}{a} \left( \vec{F} (x + a ) -\vec{F} (x)\right) \right)\)\tabularnewline
&
\(=\lim_{a \to 0}\left(\frac{1}{a} \begin{pmatrix} f_1( x + a) -f_1(x) \\ \vdots \\ f_n(x + a) -f_n(x) \end{pmatrix} \right)\)\tabularnewline
&
\(=\lim_{a \to 0}\begin{pmatrix} \frac{f_1( x + a) -f_1(x)}{a} \\ \vdots \\ \frac{f_n(x + a) - f_n(x+a)}{a} \end{pmatrix}\)\tabularnewline
&
\(=\begin{pmatrix}\lim_{a\to0}\left(\frac{f_1(x+a)-f_1(x)}{x}\right) \\ \vdots \\ \lim_{a \to 0} \left( \frac{f_n(x+a) - f_n(x)}{x} \right) \end{pmatrix}\)\tabularnewline
&
\(=\begin{pmatrix} \frac{d f_1(x)}{dx} \\ \vdots \\ \frac{d f_n(x)}{dx} \end{pmatrix}\)\tabularnewline
\bottomrule
\end{longtable}

\hypertarget{differenetiates-of-fmathbbrn-to-mathbbr}{%
\subsection[Differenetiates of
\(f:\mathbb{R}^n \to \mathbb{R}\)]{\texorpdfstring{Differenetiates
\footnote{functions resulting a scaler are not written as vectors!}of
\(f:\mathbb{R}^n \to \mathbb{R}\)}{Differenetiates of f:\textbackslash{}mathbb\{R\}\^{}n \textbackslash{}to \textbackslash{}mathbb\{R\}}}\label{differenetiates-of-fmathbbrn-to-mathbbr}}

The differentiate of \(f\) where \(f: \mathbb{R} \to \mathbb{R}^n\) is
called the gradient of \(f\) and is written \(\nabla f\)

\(\nabla f := \begin{pmatrix} f_{a}, \cdots, f_{n} \end{pmatrix}\)

\begin{quote}
Where \(f_x\) is \(\frac{\partial f(a, \cdots n)}{\partial x}\) and
\((a \cdots n)\) are the arguments of \(f\)
\end{quote}

\hypertarget{differentaites-of-vecf-mathbbrm-to-mathbbrn}{%
\subsection{\texorpdfstring{Differentaites of
\(\vec{F}: \mathbb{R}^m \to \mathbb{R}^n\)}{Differentaites of \textbackslash{}vec\{F\}: \textbackslash{}mathbb\{R\}\^{}m \textbackslash{}to \textbackslash{}mathbb\{R\}\^{}n}}\label{differentaites-of-vecf-mathbbrm-to-mathbbrn}}

\(\vec{F}( x_1 \cdots x_n)= \begin{pmatrix} f_1(x_1, \cdots x_n) \\ \vdots \\ f_m(x_1 \cdots x_n)\end{pmatrix}\)

\(\Rightarrow \vec{F}'( x_1 \cdots x_n) = \begin{pmatrix} \nabla f_1(x_1, \cdots x_n) \\ \vdots \\ \nabla f_m(x_1 \cdots x_n)\end{pmatrix}\)\hypertarget{product-rules}{%
\section{Product rules}\label{product-rules}}

\(\vec{F},\vec{G} : \mathbb{R}^n \to \mathbb{R}^m\) and
\(f:\mathbb{R}^n \to \mathbb{R}\)

\begin{itemize}
\tightlist
\item
  \textbf{(i)}
  \((\vec{F} \cdot \vec{G}) '= \vec{F}^T \vec{G}' + \vec{G}^T \vec{F}'\)
\item
  \textbf{(ii)}
  \(a (\vec{F} + b \vec{G}' = \vec{F})' + b\vec{G}' \forall a,b \in \mathbb{R}\)
\item
  \textbf{(iii)} \((f\vec{F})' = f\vec{F}' + \vec{F}f'\)
\end{itemize}

\(\vec{F}, \vec{G} : \mathbb{R} \to \mathbb{R}^3\)

\begin{itemize}
\tightlist
\item
  \textbf{(iv)} \((F \times G)' = F \times G' + F' \times G\)
\end{itemize}

The proofs are all simple evalution of the indivdiual components of each
part.\hypertarget{properties-of-the-derivatives}{%
\section{Properties of the
derivatives}\label{properties-of-the-derivatives}}

\begin{itemize}
\tightlist
\item
  \$ \nabla\$ is linear ie distrabutive over addition and commutitive
  with multiplication
\item
  \(\nabla (fg) = g \nabla f + f \nabla g\) \textbf{Chain Rule}
\item
  \$ \nabla \cdot\$ is linear over Rn to Rm functions
\item
  \$ \nabla \cdot (gF) = (\nabla g) \cdot F + g \nabla F\$
\item
  Same is true for cross product for R3 to R3
\item
  \nabla \cdot (\mathbf{F} \time \mathbf{G} =
  (\nabla \times \mathbf{F})\cdot \mathbf{G}-(\nabla \times \mathbf{G}\cdot F))
\item
  Partial derivatives are comutitive
\item
  for \(f:: \mathbf{R^3} \to mathbb[R]\) and \{F :: \mathbb{R}\^{}3
  \to \mathbb{R}\^{}3\}

  \begin{itemize}
  \tightlist
  \item
    \(\nabla \time \nabla f \equiv 0\)
  \item
    \(\nabla \cdot (\nabla \time F) =0\)
  \end{itemize}
\end{itemize}\hypertarget{tangents}{%
\section{Tangents}\label{tangents}}

\textsc{To find} the tangent of a equation.

\begin{itemize}
\item
  Take the vector equation of a parametric equation \(\vec{F}(t)\) at
  \(t_0\)
\item
  The point of the line is given by \(\vec{F}(t_0)\)
\item
  The Direction is given by the derivative of the function times some
  vector or perameter u\footnote{get better wording}
  \(T(u) = \vec{F}(t_0) + d\vec{F}(t_0 ; u)\)
  \(= \vec{F}(t) + \vec{F}' \cdot u\)
\end{itemize}Taylor series:

\(f(x) = \sum_{n=0}^\infty \frac{f^{(n)}(a)}{n!} (x -a)^n\)

\(\vec{f}(\vec{x}+\vec{h})= f(x)\) Try find another explaination or
consult\hypertarget{full-notation}{%
\section{Full Notation}\label{full-notation}}

\textbf{Definition}

\(:=\) (is defined as)

\(\nabla := \begin{pmatrix} \frac{\partial}{\partial x_1} \\ \vdots \\ \frac{\partial}{\partial x_n}\end{pmatrix}\)

\(\nabla \vec{F} := \begin{bmatrix} \frac{\partial f}{\partial x_1} \\ \vdots \\ \frac{\partial f}{\partial x_n}\end{bmatrix}\)
\textbf{This is only defined for \(\mathbb{R}^n \to \mathbb{R}\)}

\(\text{div} \vec{F} := \nabla \cdot \vec{F}\) \textbf{This is definied}
\((a)_4\)

\textbf{Laplacian}
\(\nabla ^2 f = \nabla \cdot \nabla f = \sum_{j=1}^n \frac{\partial ^2 f}{\partial x^2_j}\)

\textbf{Curl for \$\vec{F} : \mathbb{R}\^{}3 \to \mathbb{R}\^{}3}
\(\text{curl} \vec{X} = \nabla \times \vec{F}\)\end{document}