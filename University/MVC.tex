
\documentclass{tufte-handout}
\usepackage{titling}
\usepackage[parfill]{parskip}
\usepackage{verbatim}
\usepackage{mystyle}
\usepackage{natbib}
\usepackage{amsmath}
\usepackage{longtable}
\everymath{\displaystyle}
\providecommand{\tightlist}{%
  \setlength{\itemsep}{0pt}\setlength{\parskip}{0pt}}
\begin{document}
test\hypertarget{introductory-analysis}{%
\section{Introductory Analysis}\label{introductory-analysis}}

\begin{verbatim}
Tut test next week!
Sakia question must be handed in at tuts.
\end{verbatim}\hypertarget{axioms-of-the-reals}{%
\subsection{Axioms of the reals}\label{axioms-of-the-reals}}

The real numbers are a field which also has ordered and diedekin
complete

\begin{description}
\tightlist
\item[Field]
A set \(F\) of with two opperators \(+\) and \(\cdot\) such that
\(\forall f_1, f_2, f_3 \in F\)
\end{description}

\begin{itemize}
\tightlist
\item
  Addition

  \begin{itemize}
  \tightlist
  \item
    \textbf{(A1)} \((f_1+f_2)+f_3 = f_1 + (f_2 + f_3)\) Addition is
    associative
  \item
    \textbf{(A2)} \(f_1+f_2 = f_1+f_2\) Addition is comutitive
  \item
    \textbf{(A3)} \(\exists 0 | f_1 + 0 = f_1\) Indentity of addition
  \item
    \textbf{(A3)} \(\exists -f_1 | f_1 + (-f_1) = 0\)
  \end{itemize}
\item
  Multiplication

  \begin{itemize}
  \tightlist
  \item
    \textbf{(M1)} \((f_1 f_2) f_3 = f_1 (f_2 f_3)\) Multiplication is
    associative
  \item
    \textbf{(M2)} \(f_1 f_2 = f_1 f_2\) Multiplication is comutitive
  \item
    \textbf{(M3)} \(\exists 1 \neq 0 | f_1 \dot 1 = f_1\) Indentity of
    addition
  \item
    \textbf{(M4)}
    \((\forall f_1 \in F | f_1 \neq 0) \exists f_1^{-1} | f_1 ( f_1^{-1}) = 1\)
  \end{itemize}
\item
  Distrabution

  \begin{itemize}
  \tightlist
  \item
    \textbf{(D1)} \(f_1(f_2+f_3)=f_1f_2 + f_1f_2\) multiplication is
    distributive over addition
  \end{itemize}
\end{itemize}

\textsc{Axioms of comparison}

\begin{itemize}
\item
  The trichotomy property of reals. \textbf{(O1)}

  \begin{itemize}
  \tightlist
  \item
    \(\left( \forall r \in \mathbb{R} \right)\) exactly one is true
  \item
    \(a > 0 \quad a = 0 \quad a < 0\)
  \end{itemize}
\item
  \(a > 0 \wedge b > 0 \quad \to \quad a + b >0\) \textbf{(O2)}
\item
  \(a > 0 \wedge b > 0 \to ab >0\) \textbf{(O3)}
\end{itemize}

\textsc{Dieekin completeness axiom} All bounded sets have superium and
infamium \textbf{(C)}\textsc{Definitions}

\begin{quote}
\begin{itemize}
\tightlist
\item
  Let \(r_1 r_2 \in \mathbb{r}\) then \(r_1\) is called larger than
  \(r_2\) written \(r_1 \> r_2\) if \(a - b \> 0\)
\item
  \(r_1 \geq b \Leftrightarrow a \> b \wedge a = b\)
\item
  \(a \leq b \Leftrightarrow a = b \wedge a \< b\)
\end{itemize}
\end{quote}

\textsc{Some proofs}

\begin{enumerate}
\def\labelenumi{\alph{enumi})}
\setcounter{enumi}{1}
\tightlist
\item
  \(a < b \wedge b < c \to a < c\quad\) transitivty\footnote{sort of}
\end{enumerate}

Proof

\begin{quote}
\(b > a \Leftrightarrow b-a \geq 0\) \textbf{(i)}

\(b -c \geq 0\) \textbf{(ii)}

combining i and ii with previous theorem \((b-a) + c - b \geq 0\)
\(a - c \geq 0\) (addition axioms) \(c \geq 0\)
\end{quote}

\textsc{Proof} by contradiction
\(a > 0 \Leftrightarrow a^{-1} > 0 \wedge a < 0 \Leftrightarrow a^{-1} < 0\)

\begin{quote}
\textbf{Case} \(a > 0\)

\begin{quote}
\textbf{(i)} \(a \neq 0\) \textbf{Trichotomy}\\
\textbf{(ii)} \(\exists a^{-1}\) \textbf{(M3)} \(\wedge\) \textbf{(i)}

\textbf{Case} \(a^{-1} < 0\)

\begin{quote}
\begin{enumerate}
\tightlist
\item
  \(-a^{-1} > 0\) \textbf{(a)}
\item
  \(-a^{-1} \cdot a > 0\) \textbf{(O3)}
\item
  \(-1 > 0\) \textbf{Contradiction}
\end{enumerate}
\end{quote}

\textbf{Case} \(a^{-1} = 0\) \textgreater{} \#. \(1 = a \cdot a^{-1}\)
\textbf{(M3)} \textgreater{} \#. \(= 0\) \textbf{(Multiplication by
zero)} \textbf{Contradiction}

\textbf{Conculsion} \(a > 0 \Leftrightarrow a^-1 >0\)
\end{quote}

\textbf{Case} \(a < 0\)

\begin{quote}
\textbf{(i)} \(a \neq 0\) \textbf{Trichotomy}\\
\textbf{(ii)} \(\exists a^{-1}\) \textbf{(M3)} \(\wedge\) \textbf{(i)}

\textbf{Case} \(a^{-1} = 0\) \textgreater{} \#.
\(1 = a \cdot a^{-1} = 0\) \textbf{Contradiciton}
\textbf{(Multiplication by zero)} \textbf{(M3)}

\textbf{Case} \(a^{-1} > 0\) \textgreater{} \#. \(-1 = -a a^{-1} > 0\)
\end{quote}
\end{quote}

k \(1 \> 0\) \textgreater{} \(a \neq 0 \to a^2 > 0\) \textgreater{}
\(use this to show 1 > 0\)

\hypertarget{absolute-value}{%
\section{Absolute value}\label{absolute-value}}

\begin{quote}
definition \(|r| = -x \forall x < 0\) \$ = x \quad x \textgreater{} 0\$
\$\infty \nin \mathbb{R} \$ a + b \leq \textbar{} a +b \$
\end{quote}

\textsc{Properties} \(|a| = 0 \Leftrightarrow if a = 0\) * if
\(|a| = 0 \to a = 0\) contrapositive * if \(a = 0 \to |a| = 0\)
deffinition

b \textbar{}-a \textbar{} = \textbar{}-a\textbar{}\footnote{Still need
  to proove} use each three cases of dichotomy

try d and e tutorial questions

\#Trainagle inequality \(||a|+|b|| \geq |a| + |b|\)

\begin{itemize}
\tightlist
\item
  \(|a + b| \leq |a| + |b| tut\)
\item
  \(| |a| - |b| \leq |a + b|\)
\end{itemize}

proof This is wronThis is wrongg \textgreater{} \textgreater{}
\(a = a - b + b \to |a| = |a - b + b| = |a-b| + |b|\) \textgreater{}
\textgreater{} \(b = b - a + a \to |b| = |b - a + a| = |b-a| + |a|\) by
c \(|b| \leq |a - b| + |a| \to |b| \leq |b - a|\) combining multiply **
\(-1\) ** \(|a - b| \leq |a| - |b| - |a - b| \leq |a| - |b| \leq a-b\)
\textbar{} by \(d\) \(\Leftrightarrow\) triangle inequility to proove
with minus just sub in \(-b\)Glossary

\begin{description}
\tightlist
\item[Bound]
The bound of a set is a number such that all numbers in the set are less
than or greater than the set. A bound is not unique.
\item[Supremum]
The supereme or infamium of a set the least upper bound or the greatest
lower bound and is unique
\item[Maximum and miniumium]
If the superium or infaium are elements of the set than they are the max
and miuminium otherwise there is no maxium or mimiumum
\end{description}\end{document}