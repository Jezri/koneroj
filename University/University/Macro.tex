\PassOptionsToPackage{unicode=true}{hyperref} % options for packages loaded elsewhere
\PassOptionsToPackage{hyphens}{url}
%
\documentclass[,twocolumn]{article}
\usepackage{lmodern}
\usepackage{amssymb,amsmath}
\usepackage{ifxetex,ifluatex}
\usepackage{fixltx2e} % provides \textsubscript
\ifnum 0\ifxetex 1\fi\ifluatex 1\fi=0 % if pdftex
  \usepackage[T1]{fontenc}
  \usepackage[utf8]{inputenc}
  \usepackage{textcomp} % provides euro and other symbols
\else % if luatex or xelatex
  \usepackage{unicode-math}
  \defaultfontfeatures{Ligatures=TeX,Scale=MatchLowercase}
\fi
% use upquote if available, for straight quotes in verbatim environments
\IfFileExists{upquote.sty}{\usepackage{upquote}}{}
% use microtype if available
\IfFileExists{microtype.sty}{%
\usepackage[]{microtype}
\UseMicrotypeSet[protrusion]{basicmath} % disable protrusion for tt fonts
}{}
\IfFileExists{parskip.sty}{%
\usepackage{parskip}
}{% else
\setlength{\parindent}{0pt}
\setlength{\parskip}{6pt plus 2pt minus 1pt}
}
\usepackage{hyperref}
\hypersetup{
            pdftitle={Macro for exams},
            pdfauthor={Jezri Krinsky},
            pdfborder={0 0 0},
            breaklinks=true}
\urlstyle{same}  % don't use monospace font for urls
\setlength{\emergencystretch}{3em}  % prevent overfull lines
\providecommand{\tightlist}{%
  \setlength{\itemsep}{0pt}\setlength{\parskip}{0pt}}
\setcounter{secnumdepth}{0}
% Redefines (sub)paragraphs to behave more like sections
\ifx\paragraph\undefined\else
\let\oldparagraph\paragraph
\renewcommand{\paragraph}[1]{\oldparagraph{#1}\mbox{}}
\fi
\ifx\subparagraph\undefined\else
\let\oldsubparagraph\subparagraph
\renewcommand{\subparagraph}[1]{\oldsubparagraph{#1}\mbox{}}
\fi

% set default figure placement to htbp
\makeatletter
\def\fps@figure{htbp}
\makeatother


\title{Macro for exams}
\author{Jezri Krinsky}
\date{}

\begin{document}
\maketitle

Scope.

\hypertarget{questions}{%
\section{Questions}\label{questions}}

question 6 in the exam

\hypertarget{general-equations}{%
\section{General equations}\label{general-equations}}

\hypertarget{short-run}{%
\section{Short run}\label{short-run}}

\hypertarget{money-supply-formula}{%
\subsection{Money supply formula}\label{money-supply-formula}}

Demand = Demand for currency + Demand for reseves

\begin{itemize}
\tightlist
\item
  = cMd + \(\theta\)Demand for cheque deposits
\item
  = cMd + \(\theta\) (1-c)md
\item
  = \((c+\theta(1-c))Md\)
\end{itemize}

\textbf{\(Md = YL(i)\)}

\hypertarget{aggregate-demand-function.}{%
\subsection{Aggregate demand
function.}\label{aggregate-demand-function.}}

\(( a_1 + \frac{md_r}{md_y}) r = a_0 + md_0\)

\hypertarget{goods-market-equilibrium}{%
\subsubsection{Goods market
equilibrium}\label{goods-market-equilibrium}}

\(Y = \alpha_0 - a_1 r\) where
\(\alpha_0 = \frac{c_0 - c_1t_0 + i_0 +G + X-M}{1-c_1(1-t_1 )-i_y}\) and
\(a_1 = \frac{i_r}{1-c_1(1-t_1)-i_y}\)

\begin{itemize}
\tightlist
\item
  \(Y = c_0 + c_1(Y - t_0 -t_1Y) +i_0 - i_r r + i_yY + G +(X-M)\)
\end{itemize}

\hypertarget{finiancial-market-equilibrium.}{%
\subsubsection{Finiancial market
equilibrium.}\label{finiancial-market-equilibrium.}}

\(Y = \beta_1 r - \beta_0\) * \(Y = \frac{md_r r - md_0 + M_s}{md _y}\)
* \(M_s = md_0 + md_yY -md_rR\)

\hypertarget{aggregate-supply-function.}{%
\subsection{Aggregate supply
function.}\label{aggregate-supply-function.}}

\hypertarget{general-formula}{%
\subsubsection{General Formula}\label{general-formula}}

\begin{itemize}
\tightlist
\item
  \(P_e F(u,z) = P \frac{1}{1+m}\)
\item
  \(P = P_e(1+m)F(u,z)\) \textbf{Solving the production function for Y,
  nd then solving for p}
\end{itemize}

\hypertarget{wage-setting}{%
\subsubsection{Wage setting}\label{wage-setting}}

\begin{itemize}
\tightlist
\item
  \(W = P_e F(u,z)\)
\end{itemize}

\hypertarget{price-setting}{%
\subsubsection{Price setting}\label{price-setting}}

\begin{itemize}
\tightlist
\item
  \(\frac{W}{P} =\frac{1}{1+m}\)
\end{itemize}

\hypertarget{medium-run}{%
\section{Medium Run}\label{medium-run}}

Start by linking the two markets

\begin{itemize}
\tightlist
\item
  \(P = P_e(1+m) F(1-\frac{Y}{L}z)\)
\end{itemize}

\hypertarget{medium-islm-or-aggregate-deamnd-with-no-expected-inflation}{%
\subsection{Medium ISLM or aggregate deamnd with no expected
inflation}\label{medium-islm-or-aggregate-deamnd-with-no-expected-inflation}}

Even in the short run price rise can be considered as moving the lm
curve inwards in the ad as model.

\hypertarget{a-chnge-in-monetary-stokes.}{%
\subsubsection{A chnge in monetary
stokes.}\label{a-chnge-in-monetary-stokes.}}

Eventual price change so thier is no change in real money suply. this
means intrest returns to its prechange level

Replace money supply with money supply over price levels. As over the
long run prices rise reducing Money supply which increases intrest rates
at any given level of employment moving the LM curve to the left.

\hypertarget{section}{%
\subsection{}\label{section}}

Prices = expected Prices \#\# Changes in price of other inputs(natural
reasoureses)

Change the value m and assume no change in the aggregrate demand

\hypertarget{phlips-curve-and-medium-run-with-expected-inflation}{%
\section{Phlips curve and medium run with expected
inflation}\label{phlips-curve-and-medium-run-with-expected-inflation}}

Start with the aggregate supply relation

\[P_e F(u,z) = P \frac{1}{1+m}\]

Than use a linear function for \(F\)

\[F = 1-\alpha u + z\]

This solves to give

\[\pi = \pi_e + m + z -\alpha u\]

\hypertarget{original-philips-curve}{%
\subsubsection{Original philips curve}\label{original-philips-curve}}

Assume pi\_e is always zero as inflation was not 2

\hypertarget{augmented-philips-curve}{%
\subsubsection{Augmented Philips curve}\label{augmented-philips-curve}}

Replace \(\pi_e\) with \(\pi_t-1 \theta\)

If we replace \(\theta\) with one we get

\[\Delta pi = m + z -\alpha u\]

\hypertarget{natural-rate-of-unemployment}{%
\paragraph{Natural rate of
unemployment}\label{natural-rate-of-unemployment}}

Gives a constant inflation rate and assume \(\theta = 1\)
\(U_n = \frac{m+z}{\alpha}\)

Plugining this back into the equation we get

\(\Delta \pi = -\alpha (u_t - u_n)\)

\hypertarget{nutrality-of-money}{%
\paragraph{Nutrality of money}\label{nutrality-of-money}}

A natural rate of unemployment implies a natural rate of output
irrespective of money supply.

\hypertarget{wage-indexation}{%
\paragraph{Wage indexation}\label{wage-indexation}}

Some wages are indexed directly on current inflation.

\(\pi_t = \lambda pi_t + (\lambda-1) pi_t-1 + m + z - \alpha u\)

This gives

\(\delta \pi = -\frac{1}{1-\lambda}{u_t - u_n}\)

when wage indexation is common small changes in employment yeiled large
changes in inflation

\hypertarget{okun-law}{%
\subsubsection{Okun law}\label{okun-law}}

\(\Delta u = -\alpha(g - \beta)\) where g is growth rate. becouse output
and employment are not infact one to one becouse of labour hording.

The natural rate of growth is one with no change in unemployment.

\hypertarget{aggregate-demand-relation}{%
\subsection{Aggregate demand relation}\label{aggregate-demand-relation}}

Take the function \(Y = Y(\frac{M}{P},G,T)\)

and replace it with

\(Y = \gamma \frac{M}{P}\)

Solving this gives that the real output growth rate is the nominal
output growth rate - the inflation rate

\hypertarget{lucas-crituque}{%
\subsection{Lucas crituque}\label{lucas-crituque}}

If \(\pi_e\) is based on credible policy declaration to a larger extent
rather than only on previous inflation the cost of disinflation is
reduced.

\textbf{Point year of execss inflation} the difference between the
actual and the naturl unemployment rate of 1 percent for one year

\hypertarget{summary}{%
\section{Summary}\label{summary}}

\begin{itemize}
\tightlist
\item
  \(ut - u_{t-1} = -a(g_{yt}-g_{yn})\) \textbf{Okuns law}
\item
  \(\pi_t - \pi_{t-1} = -{u_t-un}\) \textbf{Philips curve}
\item
  \(g_{yt} = g_{mt} -\pi_t\) \textbf{Aggregate demand}
\end{itemize}

\end{document}
