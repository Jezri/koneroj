
\documentclass{tufte-book}
\usepackage{titling}
\usepackage{fancyhdr}
\pagestyle{fancy}
\fancyhf{}
\rhead{\thesection}
\lhead{}
\rfoot{\color{titlecolor}{ Page \thepage}}
\usepackage[parfill]{parskip}
\usepackage{verbatim}
\usepackage{mystyle}
\usepackage{natbib}
\usepackage{amsmath}
\usepackage{longtable}
\everymath{\displaystyle}
\providecommand{\tightlist}{%
  \setlength{\itemsep}{0pt}\setlength{\parskip}{0pt}}
\begin{document}
\hypertarget{macro-economics}{%
\section{Macro Economics}\label{macro-economics}}

\hypertarget{gdp}{%
\subsection{GDP}\label{gdp}}

\begin{description}
\item[GDP(Gross Domestic product)]
The Total amount of goods and services produced within an economy in a
given year \footnote{There are three ways of culculating this

  \begin{itemize}
  \item
    Expenditure : This must only include expenditure on goods and
    services produced within the economy (no imports, and no goods
    produced in a previous year)
  \item
    Income : This must only use income obtained by selling goods and
    services (no transfer payments)
  \item
    Output
  \end{itemize}}
\end{description}

\hypertarget{gdp-composition}{%
\section{GDP composition}\label{gdp-composition}}

To measure the GDP\footnote{GDP and total demand(Z) are used
  interchangably} it is simplest to measure the amount spent on goods
and services and then subtract the part of that which is spent on goods
and services produced outside the economy (imports) or before the given
year (invetories). Finaly goods not bought in the bought elsewhere
(exports) or stored for the future are added.\footnote{Exports and
  inventories are ignored in the begining part of the course}

\begin{itemize}
\tightlist
\item
  Consumption(C): The goods and services purchased by consumers
\item
  Investment(I): The sum of

  \begin{itemize}
  \tightlist
  \item
    no-resedential investment: Capital equipment and land bought by
    firms
  \item
    resedntial investment: Housing bought by consumers
  \end{itemize}
\item
  Goverment spending(G): The amount the goverment spendings buying goods
  and services from firms and employing workers. (goverment tranfers are
  not payments for work done and are not included)
\item
  Net exports (X-I): The total amount of exports minus imports.
\item
  Net inventory build up
\end{itemize}

This brings us to the equation \(Z = C + I + G\)

\hypertarget{consumption}{%
\subsection{Consumption}\label{consumption}}

Consumption is a function of disposable income \footnote{income minus
  taxation} (\(Y_D\)) \[ C(Y_D) \]

\hypertarget{unemployment}{%
\subsection{Unemployment}\label{unemployment}}

\hypertarget{inflation}{%
\subsection{Inflation}\label{inflation}}

\hypertarget{philips-curve}{%
\subsection{Philips curve}\label{philips-curve}}\hypertarget{islm-model}{%
\section{ISLM model}\label{islm-model}}

The ISLM model models the equilibrium for income and interest rates, the
quintity of money supplied and consumption.

This is represented by the intersenctoin of the IS curve which shows the
resultant intrest rates for all equilibriums between money demand which
is a function of income and money supply, and the LM curve which
represents all the points of equilibrium (where consumption is equil to
output and no inventories are built up or depleted) and consuption
(which contains investment which is a function of interest rates)\hypertarget{financial-markets}{%
\section{Financial Markets}\label{financial-markets}}

\hypertarget{the-demand-for-money}{%
\subsection{The Demand for money}\label{the-demand-for-money}}

Is the demand for liquidity and comes from people wanting to make
purchases with that money.

\hypertarget{supply-of-money}{%
\subsection{Supply of money}\label{supply-of-money}}

The supply of money is modeled as a constant \(M\) and supplied by the
centerall bank

\hypertarget{monatary-policy-and-open-market-operations-58}{%
\subsection{Monatary policy and open market operations
58}\label{monatary-policy-and-open-market-operations-58}}\end{document}