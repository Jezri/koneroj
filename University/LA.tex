
\documentclass{tufte-handout}
\usepackage{titling}
\usepackage[parfill]{parskip}
\usepackage{verbatim}
\usepackage{mystyle}
\usepackage{natbib}
\usepackage{amsmath}
\usepackage{longtable}
\everymath{\displaystyle}
\providecommand{\tightlist}{%
  \setlength{\itemsep}{0pt}\setlength{\parskip}{0pt}}
\begin{document}
\hypertarget{linear-algebra}{%
\section{Linear algebra}\label{linear-algebra}}

\begin{description}
\tightlist
\item[Field]
A set \(F\) of with two opperators \(+\) and \(\cdot\) such that
\(\forall f_1, f_2, f_3 \in F\)
\end{description}

\begin{itemize}
\tightlist
\item
  Addiition

  \begin{itemize}
  \tightlist
  \item
    \((f_1+f_2)+f_3 = f_1 + (f_2 + f_3)\) Addition is associative
  \item
    \(f_1+f_2 = f_1+f_2\) Addition is comutitive
  \item
    \(\exists 0 | f_1 + 0 = f_1\) Indentity of addition
  \item
    \(\exists -f_1 | f_1 + (-f_1) = 0\)
  \end{itemize}
\item
  Multiplication

  \begin{itemize}
  \tightlist
  \item
    \((f_1 f_2) f_3 = f_1 (f_2 f_3)\) Multiplication is associative
  \item
    \(f_1 f_2 = f_1 f_2\) Multiplication is comutitive
  \item
    \(\exists 1 \neq 0 | f_1 \dot 1 = f_1\) Indentity of addition
  \item
    \((\forall f_1 \in F | f_1 \neq 0) \exists f_1^ {-1} | f_1 ( f_1^{-1}) = 1\)
  \end{itemize}
\item
  Distrabution

  \begin{itemize}
  \tightlist
  \item
    \(f_1(f_2+f_3)=f_1f_2 + f_1f_2\) multiplication is distributive over
    addition
  \end{itemize}
\end{itemize}\hypertarget{properties-of-vector-spaces}{%
\subsection{Properties of vector
spaces}\label{properties-of-vector-spaces}}

\begin{itemize}
\tightlist
\item
  \(0 \cdot \vec{a} = \vec{0}\)
\end{itemize}

\begin{quote}
Proof

\(\quad 0 \cdot \vec{a} = (0 + 0) \cdot \vec{a} \quad \text{A3}\)

\(\quad 0 \cdot \vec{a} = 0 \cdot \vec{a} + 0 \cdot \vec{a} \quad \text{Dist}\)

\(\quad 0 \cdot \vec{a} - (0 \cdot \vec{a}) = 0 \cdot \vec{a} + 0 \cdot \vec{a} - ( 0 \cdot \vec{a})\)

\(0 0 = 0 \cdot \vec{a} \quad \text{A4}\)
\end{quote}

\begin{itemize}
\tightlist
\item
  \((-1) \vec{a} = -a\)
\end{itemize}

\begin{quote}
Proof

\((-1) \vec{a} = -a \quad \Leftrightarrow \quad (-1) \vec{a} +1 \vec{a} = 0 \quad \text{M3}\)

\(\quad 1 \vec{a} + (-1) \vec{a} \quad \text{M3}\)

\(=(1 - 1) \cdot \vec{a} \quad \text{Dist}\)

\(=0 \cdot \vec{a}\)

\(=\vec{0}\)
\end{quote}

\begin{itemize}
\tightlist
\item
  \(\lambda \vec{0} = 0\)
\end{itemize}

\begin{quote}
Proof

Find this proof
\end{quote}

\begin{itemize}
\tightlist
\item
  \(\lambda \vec{a} = \vec{0} \Leftrightarrow \lambda = 0 \wedge \vec{a}=0\)
\end{itemize}

\begin{quote}
Find proof
\end{quote}\hypertarget{linear-dependency}{%
\section{Linear Dependency}\label{linear-dependency}}

Let \(V\) be a vector space over \(F\) A set of vectors
\(\{\vec{\alpha _i} \to \vec{\alpha _n}\}\) if there are no
nontrivail.\footnote{\begin{description}
  \tightlist
  \item[trivial]
  \(\forall i a_i = 0\)
  \item[nontrivial]
  \(\exist i | a_i \neq 0\)
  \end{description}} solutions to the equation
\(\sum_{i=1}^n a_i \cdot \vec{\alpha_i}=0\) This is equivelent to saying
\(\mathbf{A}\vec{x}= 0\) has only trivial solutions

This implies that elemtry row operations to get a reduced row echelon
form will result in no zero rows.

The contrapositive stamament is that the existance of a nontrivial
solution implies that the system is the system is linearly dependent.\hypertarget{propositions}{%
\section{Propositions}\label{propositions}}

\textbf{Proposion} For a system of vec \{a1 .. aN\} the following
statents are equiv

\begin{itemize}
\tightlist
\item
  \{a\_1 to a\_n\} is linearly dependent
\item
  \exist a\_m \textbar{} a\_m is a linear combination of the remaining
  vectors
\item
  at least one vectors can be a\_1 \ldots{} a\_n is a linear combination
  of the preseading vectors.\footnote{Why is this true}
\end{itemize}

\footnote{Order is important} proove mulitple statment are equivenlt
using a cycle (a \to b \to c \to a) \textgreater{} Proof \textgreater{}
\textgreater{} 1( \to 3 \textgreater{} \textgreater{} \exists a
nontiravlai combination of coefficient such that \textbf{A}
\(\cdot \vec{a} = \vec{0}\) \textgreater{} \textgreater{} take the last
vector of a nontriavial soltion without a zero coeefiecint
\(\alpha j \neq 0\) \textgreater{} \textgreater{}
\(a_j = a_1(\alpha_1)(\alpha_j)a_1 ...\)

3 \to 2 \(a_j = \beta a_2 + b i-1 + b i + 1 +b)n = b_n = 0\)
\(a_i = \beta b-1 a_1 + b_1 a-1 + b_n a_n\)

\(2 \to a \beta \alpha + \beta \alpha = 0\) non trivial becouse beta j
is not zero full proof\hypertarget{span}{%
\section{Span}\label{span}}

Let v be a vector space over a field f and let s be \subset V the span
of S denoted \textless{} S\textgreater{} is the subset of V consisting
of all vectors whcih can be represented as a linear combination of
vectors from s. That is alpha 1 a1 + alpha n a n from S \{\apha\} \in F

Proove that 0 \in <S> Trivial combination S \in <S> a = 1.a iii)

\hypertarget{span-2}{%
\section{Span 2}\label{span-2}}

The set of all vectors from v that can be expresed by linear
combinations of the vectors of s \(<s> = \sum \alpha _i a_i\)

\begin{itemize}
\tightlist
\item
  The zero vector is always in the span
\item
  s is a subset of s
\item
  s is a subset of t implies the span of s is a subset of t
\item
  the span of the span of s is the span of s

  \begin{itemize}
  \tightlist
  \item
    Proof
  \item
    \(<s>\) is a subset of \(<<s>>\) \footnote{Now proove cthe converse}
  \item
    \(c=\beta_1b_1 + \cdots + \beta_m b_m \in <s> a_1 a_n\in <s>\)
  \item
    \(b_i = \alpha_1a_1 + \cdots \alpha_ma_m\)
  \item
    \(c=(\Beta_1\alpha_{11}+\cdots \Beta_m \alpha_{m1}a_1 \cdots + (\beta_m \alpha_{1m}\cdots)\)
  \end{itemize}
\item
  if we add to s a vector which is a linear combination of vectors from
  s or remove from s a vector which is a linear combination of the
  remaining vector from s then the span of s remains the same

  \begin{itemize}
  \tightlist
  \item
    a collarary of 4
  \end{itemize}
\item
  A set of vectors is linearly independent if and only if the system
  without the last vector is linearly independent and the last vector is
  not in the span of the previous system

  \begin{itemize}
  \tightlist
  \item
    Proof
  \item
    suppose the smaller system is linearly dependent
  \item
    Then the new system will be linearly dependent by simply adding a
    zero coeeficient to the last vector becouse there will be a none
    zero coeeficient in the previou system so the solution to the new
    system will also be nontrivial
  \item
    To see the escond stament
  \item
    Assume on the contrary that the new vector belongs the the span of
    the previous set of vectors than it can be shown easily that there
    exist a none trivial solution to the new system.
  \item
    Sufficiency
  \item
    Assume on the contrary the the new set is linearly dependent
  \item
    There is a nontrivial combination for the new system which equale
    \(\Theta\) such that \(\alpha_{n+1} \neq0\) otherwise only
    nontrivial solutions would exist!
  \item
    But this contradicts with the fact that the new vector is not in the
    span of the previous vectors

    \begin{itemize}
    \tightlist
    \item
      But than a(n+1) = \(\frac{\alpha_1}{\alpha_{n+1}}a_1 \codts\)
      which meens that an is an in the span of the previous vectors
    \end{itemize}
  \end{itemize}
\end{itemize}Basis :

\begin{itemize}
\tightlist
\item
  Let v be a vector space over a fiel \textbf{F}
\item
  A system of vectors is a basis of v if

  \begin{itemize}
  \tightlist
  \item
    it is linearly independent
  \item
    the span of the vectors is v that is every vector a from v can be
    written as a linear combination of those vectors\footnote{The
      coefficents called cordinates can be determined uniquely}
  \end{itemize}
\end{itemize}

Let S be a finite subset of V whcih spans the whole space. Then there is
a basis of for V contianed in s.

\begin{quote}
Proof \textgreater{} \textgreater{} It suffice to choose a linearly
indepent system a\_1 to a\_n a subset in s whose span contains s
\textgreater{} Becouse if
\(<a_1 \cdots a_n>=<<a_1 \codt a_n>> \subset <S>\) \textgreater{} If
\in S , we are done \textgreater{} Otherwise we can pick a3 from s -a1 -
a2 and in \(\leq |S\) we obtain the required system
\end{quote}\end{document}